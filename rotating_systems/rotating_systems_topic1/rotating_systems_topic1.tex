% 
% Lecture Template for ME3050 -  Dynamics Modeling and Controls - Tennessee Technological University
%
% Spring 2020 - Summer 2020
% Tristan Hill, May 07, 2020 - June 12, 2020
% Module 5 - Rotating Systems
% Topic 1 - The Dynamics of Rotation
%

\documentclass{beamer}                         % for presentation (has nav buttons at bottom)
%\documentclass[handout]{beamer}  % for handout 
\usepackage{beamerthemesplit}
\usepackage{amsmath}
\usepackage{listings}
\usepackage{multicol}
\usepackage{framed}

\beamertemplateballitem

% custom colors
\definecolor{TTUpurple}{rgb}{0.3098, 0.1607, 0.5176} % TTU Purple (primary)
\definecolor{TTUgold}{rgb}{1.0000, 0.8666, 0.0000} % TTU Gold (primary) 
\definecolor{mygray}{rgb}{.6, .6, .6}
\definecolor{mypurple}{rgb}{0.6,0.1961,0.8}
\definecolor{mybrown}{rgb}{0.5451,0.2706,0.0745}
\definecolor{mygreen}{rgb}{0, .39, 0}
\definecolor{mypink}{rgb}{0.9960, 0, 0.9960}

% color commands
\newcommand{\R}{\color{red}}
\newcommand{\B}{\color{blue}}
\newcommand{\BR}{\color{mybrown}}
\newcommand{\K}{\color{black}}
\newcommand{\G}{\color{mygreen}}
\newcommand{\PR}{\color{mypurple}}
\newcommand{\PN}{\color{mypink}}
\newcommand{\OR}{\color{TTU}}
\newcommand{\GD}{\color{TTUgold}}


\setbeamercolor{palette primary}{bg=TTUpurple,fg=TTUgold}
\setbeamercolor{palette secondary}{bg=black,fg=TTUgold}
\setbeamercolor{palette tertiary}{bg=black,fg=TTUpurple}
\setbeamercolor{palette quaternary}{bg=TTUgold,fg=black}
\setbeamercolor{structure}{fg=TTUpurple} % itemize, enumerate, etc
\setbeamercolor{section in toc}{fg=TTUpurple} % TOC sections

%\usefonttheme{professionalfonts}

\newcommand{\Lagr}{\mathcal{L}} % lagrangian

\newcommand{\hspcu}{\underline{\hspace{20mm}}} % large horizontal space w underline
\newcommand{\vspccc}{\vspace{6mm}\\} % large vertical space
\newcommand{\vspcc}{\vspace{4mm}\\}   % medium vertical space
\newcommand{\vspc}{\vspace{2mm}\\}     % small vertical space

\newcommand{\hspcccc}{\hspace{10mm}} % large horizontal space
\newcommand{\hspccc}{\hspace{6mm}} % large horizontal space
\newcommand{\hspcc}{\hspace{4mm}}   % medium horizontal space
\newcommand{\hspc}{\hspace{2mm}}     % small horizontal space

\newcommand{\eqscl}{0.9}     % small horizontal space


\author{ME3050 - Dynamics Modeling and Controls} % original formatting from Mike Renfro, September 21, 2004

\newcommand{\MNUM}{5\hspace{2mm}} % Module number
\newcommand{\TNUM}{1\hspace{2mm}} % Topic number 
\newcommand{\moduletitle}{Rotation Systems }
\newcommand{\topictitle}{The Dynamics of Rotation} 

\newcommand{\sectiontitleI}{Newton's Second in Rotation}
\newcommand{\sectiontitleII}{Fixed Axis Rotation}
\newcommand{\sectiontitleIII}{Energy of Rotation}
\newcommand{\sectiontitleIV}{Engineering Applications}

% custom box
\newsavebox{\mybox}

\title{Module \MNUM - \moduletitle}

\date{Mechanical Engineering\vspc Tennessee Technological University}

\begin{document}

\lstset{language=MATLAB,basicstyle=\ttfamily\small,showstringspaces=false}

\frame{\titlepage \center\begin{framed}\Large \textbf{Topic \TNUM - \topictitle}\end{framed} \vspace{5mm}}

% Section 0: Outline
\frame{

\large \textbf{Topic \TNUM - \topictitle} \vspace{3mm}\\

\begin{itemize}

	\item \sectiontitleI		\vspc % Section I
	\item \sectiontitleII 	\vspc % Section II
	\item \sectiontitleIII 	\vspc %Section III
	\item \sectiontitleIV 	\vspc %Section IV

\end{itemize}

}

% Section I:
\section{\sectiontitleI}

\frame{
\frametitle{\sectiontitleI}

In a system involving rotation Newton's Second Law equates the mass moment of inertia to the angular acceleration of a rigid body. \vspc

Also, the moment of inertia of an object depends on not only its mass but the geometry with respect to the center of rotation and can be thought of a the rotational equivalent of mass. \vspc

\scalebox{1}{$\Sigma M = I \alpha$}

}

\frame{
\frametitle{\sectiontitleI}

%\vspace{5mm}
%{\tiny Text: \href{https://en.wikipedia.org/wiki/Conservation_of_energy}{Wikipedia} }

}

% Section II:
\section{\sectiontitleII}

\frame{
\frametitle{\sectiontitleII}


\vspace{1mm}
{\tiny Images: T.Hill}

}

% Section III:
\section{\sectiontitleIII}

\frame{
\frametitle{\sectiontitleIII}

% Section III:
\section{\sectiontitleIII}

\frame{
\frametitle{\sectiontitleIII}


}
}

% Section IV:
\section{\sectiontitleIV}

\frame{
\frametitle{\sectiontitleIV}


}
	
\end{document}





