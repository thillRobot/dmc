% ME3050 -  Dynamics Modeling and Controls - Tennessee Technological University
% Tristan Hill - Spring 2020 - Summer 2020 - Spring 2022
% Dynamics Modeling and Controls
% Lecture Module - Fluid Systems - Topic 1  - Basic Concepts

% Document settings

\documentclass{beamer}                  % for presentation ?
%\documentclass[handout]{beamer}  % for handout ?

\usepackage{/home/thill/Documents/lectures/dmc_lectures/dmc_lectures}

\newcommand{\MNUM}{2\hspace{2mm}} % Module number
\newcommand{\TNUM}{1\hspace{2mm}} % Topic number 
\newcommand{\moduletitle}{Fluid Systems} % Titles and Stuff
\newcommand{\topictitle}{Basic Concepts} 

\newcommand{\sectiontitleI}{Definitions and Concepts} % More Titles and Stuff
\newcommand{\sectiontitleII}{Conservation of Mass}
\newcommand{\sectiontitleIII}{Fluid Capacitance and Resistance}
\newcommand{\sectiontitleIV}{Dynamic Models of Hydraulic Systems}
\newcommand{\sectiontitleV}{Examples}

\author{ME3050 - Dynamic Modeling and Controls}
\title{Lecture Module - \moduletitle}
\date{Mechanical Engineering\vspc Tennessee Technological University}

\begin{document}
	
	\lstset{language=MATLAB,basicstyle=\ttfamily\small,showstringspaces=false}
	
	\frame{\titlepage \center\begin{framed}\Large \textbf{Topic \TNUM - \topictitle}\end{framed} \vspace{5mm}}
	
	% Section 0 - Outline
\frame{
	
	\large \textbf{\moduletitle} \vspace{3mm}\\
	
	\begin{itemize}
	
		\item \hyperlink{sectionI}{\sectiontitleI} \vspc % Section I
		\item \hyperlink{sectionII}{\sectiontitleII} \vspc % Section II
		\item \hyperlink{sectionIII}{\sectiontitleIII} \vspc %Section III
		\item \hyperlink{sectionIV}{\sectiontitleIV} \vspc %Section IV	
		\item \hyperlink{sectionV}{\sectiontitleV} \vspc %Section V
	
	\end{itemize}

	\btVFill
	\tiny{System Dynamics, Palm, 4$^{th}$}	

}

% Section I
\section{\sectiontitleI}

	% Section I - Frame I
	\begin{frame}[label=sectionI] \small
		\frametitle{\sectiontitleI}

		{\it A fluid system uses one or more fluids to achieve its purpose.}

		Examples:
		\begin{itemize}
			\item Fluid Damper in Suspension (Shock Absorber)
			\item Hydraulic Front Loader (Tractor)
			\item Fuel Delivery System 
			\item HVAC System
		\end{itemize}

	\end{frame}	

    	% Section I - Frame II
	\begin{frame}[label=sectionI] \small
		\frametitle{\sectiontitleI}
		Fluid systems can be categorized as either hydraulic or pneumatic systems. \vspace{5mm}\\

		{\bf Hydraulic:}\vspace{10mm}\\

		{\bf Pneumatic:}\vspace{10mm}\\ 

		
		For incompressible fluids, the conservation of mass becomes the: \vspace{2mm}\\
		\underline{\hspace{40mm}} of \underline{\hspace{40mm}}.

		\btVFill
		\tiny{System Dynamics, Palm, 4$^{th}$}	

	\end{frame}

	%\section{\sectiontitleII}	
	
	% Section I - Frame III
	\begin{frame}[label=sectionI] \small
		\frametitle{\sectiontitleI}
		%\bigskip
		

		\btVFill
		\tiny{System Dynamics, Palm, 4$^{th}$}		
		
	\end{frame}

% Section II
\section{\sectiontitleII}

	% Section II - Frame I
	\begin{frame}[label=sectionII,containsverbatim] \small
%		\frametitle{\sectiontitleII}
		The mass density and the volume flow rate can be used to find the volume flow rate.
		\[ q_m=\rho q_v \]
		
	\end{frame}

	%Section II - Frame II
	\begin{frame} \small
		\frametitle{\sectiontitleII}
			
		The conservation of mass is stated below.

		\[\dot{m}=q_{mi}-q_{mo}\]

		If the fluid is incompressible, this relation can be re-written as the conservation of volume.

		\[\dot{m}=\rho \dot{V} \implies q_{mi}=\rho q_{vi} \hspace{2mm} and \hspace{2mm} q_{mo}=\rho q_{vo}\]
		\[\rho\dot{V}=\rho q_{vi}\ -\rho q_{vo}\]
		\[\dot{V}=q_{vi}-q_{vo}\]

	\end{frame}	

% Section III
\section{\sectiontitleIII}

	% Section III - Frame I
	\begin{frame}[label=sectionIII] \small
		\frametitle{\sectiontitleIII}	
		\bigskip

		Fluid systems can be compared to equivalent electrical systems. \vspace{3mm}\\ 

		{\bf Analogous Quantities} \vspace{3mm}\\ 
		\begin{tabular}{|c|c|}
			Fluid Mass, $m$ & Charge, $Q$ \\
			Mass Flow Rate, $q_m$ & Current, $i$ \\	
			Pressure, $p$ & Voltage, $v$ \\
			Fluid (linear) Resistance , $R$ & Electrical Resistance, $R$ \\
			$R=p/q_m$ & $R=v/i$ \\
			Fluid Capacitance, $C$ & Electrical Capacitance, $C$ \\
			$C=m/p$ & $ C=Q/v$ \\
			Fluid inertance, $I$ & Electrical Inductance, $L$ \\
			$I=p/\left( \frac{dq_m}{dt}\right)$ & $L=v/\left( \frac{di}{dt}\right)$ \\

		\end{tabular}

		\btVFill
		\tiny{System Dynamics, Palm, 4$^{th}$}	
	\end{frame}	
	
	%Section III - Frame II
	\begin{frame} \small
		\frametitle{\sectiontitleIII}
		\bigskip

		{\bf Fluid Resistance} is the relation between pressure and mass flow rate. \vspace{15mm}\\

		{\bf Fluid Capacitance} is the relation between pressure and \vspace{3mm}\\ 
		\underline{\hspace{20mm}} \hspace{2mm} \underline{\hspace{20mm}}.


		\btVFill
	\end{frame}	

	%Section III - Frame III
	\begin{frame} \small
		\frametitle{\sectiontitleIII}

		Can you relate Kirchoff's laws to a fluid system? \vspace {10mm}\\

		The \underline{\hspace{20mm}} law is analogous to Kirchoff's voltage law (KVL).  \vspace {10mm}\\


		The \underline{\hspace{20mm}} law is analogous to Kirchoff's current law (KVL).

			
	\end{frame}	

% Section IV
\section{\sectiontitleIV}

	% Section IV - Frame I
	\begin{frame}[label=sectionIV] \small
		\frametitle{\sectiontitleIV}
		
	\end{frame}	

% Section V
\section{\sectiontitleV}

	% Section V - Frame I
	\begin{frame}[label=sectionV] \small
		\frametitle{\sectiontitleV}
		
	\end{frame}	

\end{document}



