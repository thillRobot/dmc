% Lecture Template for ME3001-001-Tristan Hill - Spring 2017 - Fall 2017
% 
% Mechanical Engineering Analysis with MATLAB
%
% Introduction to Analysis

% Document settings
\documentclass[11pt]{article}
\usepackage[margin=1in]{geometry}
\usepackage[pdftex]{graphicx}
\usepackage{multirow}
\usepackage{setspace}
\usepackage{hyperref}
\usepackage{color,soul}
\usepackage{fancyvrb}
\usepackage{framed}
%\usepackage{wasysym}
\usepackage{multicol}
\usepackage{amssymb }
\usepackage{amsmath}



\pagestyle{plain}
\setlength\parindent{0pt}
\hypersetup{
    bookmarks=true,         % show bookmarks bar?
    unicode=false,          % non-Latin characters in Acrobat’s bookmarks
    pdftoolbar=true,        % show Acrobat’s toolbar?
    pdfmenubar=true,        % show Acrobat’s menu?
    pdffitwindow=false,     % window fit to page when opened
    pdfstartview={FitH},    % fits the width of the page to the window
    pdftitle={My title},    % title
    pdfauthor={Author},     % author
    pdfsubject={Subject},   % subject of the document
    pdfcreator={Creator},   % creator of the document
    pdfproducer={Producer}, % producer of the document
    pdfkeywords={keyword1} {key2} {key3}, % list of keywords
    pdfnewwindow=true,      % links in new window
    colorlinks=true,       % false: boxed links; true: colored links
    linkcolor=red,          % color of internal links (change box color with linkbordercolor)
    citecolor=green,        % color of links to bibliography
    filecolor=magenta,      % color of file links
    urlcolor=blue           % color of external links
}

% assignment number 
\newcommand{\NUM}{1 } 
\newcommand{\VSpaceSize}{2mm} 
\newcommand{\HSpaceSize}{2mm} 

%newcommand{\Lagr}{\mathcal{L}}

\DeclareMathOperator{\Lagr}{\mathcal{L}}


\definecolor{mygray}{rgb}{.6, .6, .6}

\setulcolor{red} 
\setstcolor{green} 
\sethlcolor{mygray} 

\begin{document}

\textbf{ \LARGE ME 3050 Lecture - Laplace Transform and Properties} \vspace{3mm}\\
\textbf{ \hspace*{5mm}Tristan W. Hill - Tennessee Technological University - Spring 2020 } \vspace{3mm}\\

\Large
\begin{itemize}

\item \textbf{ \LARGE The Laplace Transform is an Integral Transform: } \\

Given a function $x(t)$ in the time domain where $t\geq0$, \\ the Laplace Transform is defined as follows: \\

\scalebox{1.25}{$X(s)=\Lagr{\{x(t)\}}=\displaystyle\int_0^{\infty} x(t)e^{-st}dt$}\\

And its inverse is similarly defined as: \\

 \scalebox{1.25}{$\Lagr^{-1}{\{X(s)\}}=x(t)$}\\

The Laplace Domain variable $s$ is a complex number: \scalebox{1.5}{$s=\sigma+j\omega$} \\

It is useful to find the laplace transform of the derivative of a function: \\

\scalebox{1.25}{$\Lagr{\{\frac{d}{dt}(x(t))\} }=\Lagr{\{\dot{x}(t)\}} =s\Lagr{\{x(t)\} }-x(t=0)$} \\

\scalebox{1.25}{$\hspace{52mm} =sX(s)-x(t=0)$} \\

\scalebox{1.25}{$\hspace{52mm} =sX(s)-x_0$} \\\\

\scalebox{1.25}{$\Lagr{\{\frac{d^2}{dt^2}(x(t))\} }=\Lagr{\{\ddot{x}(t)\}} = s^2\Lagr{\{x(t)\} }-sx(t=0)-\dot{x}(t=0)$}\\

\scalebox{1.25}{$\hspace{52mm} = s^2X(s)-sx(t=0)-\dot{x}(t=0)$} \\

\scalebox{1.25}{$\hspace{52mm} = s^2X(s)-sx_0-\dot{x}_0$}

\end{itemize}

\end{document}



