% ME3050 -  Dynamics Modeling and Controls - Tennessee Technological University
% Tristan Hill - Spring 2020 - Summer 2020 - Spring 2022
% Dynamics Modeling and Controls
% Lecture Module - Laplace Transforms - Topic 3  - Partial Fraction Decomposition 

% Document settings

%\documentclass{beamer}                  % for presentation ?
\documentclass[handout]{beamer}  % for handout ?

\usepackage{/home/thill/Documents/lectures/dmc_lectures/dmc_lectures}

%\newcommand{\MNUM}{10\hspace{2mm}} % Module number
\newcommand{\TNUM}{3\hspace{2mm}} % Topic number 
\newcommand{\moduletitle}{The Laplace Transform} % Titles and Stuff
\newcommand{\topictitle}{Partial Fraction Decomposition} 

\newcommand{\sectiontitleI}{General Polynomial Form} % More Titles and Stuff
\newcommand{\sectiontitleII}{Case 1 - Distinct Roots}
\newcommand{\sectiontitleIII}{Case 2 - Repeated Roots}
\newcommand{\sectiontitleIV}{Special Case - Complex Roots}

\author{ME3050 - Dynamic Modeling and Controls}
\title{Lecture Module - \moduletitle}
\date{Mechanical Engineering\vspc Tennessee Technological University}

\begin{document}
	
	\lstset{language=MATLAB,basicstyle=\ttfamily\small,showstringspaces=false}
	
	\frame{\titlepage \center\begin{framed}\Large \textbf{Topic \TNUM - \topictitle}\end{framed} \vspace{5mm}}
	
	% Section 0 - Outline
	\frame{
		
		\large \textbf{Topic \TNUM - \topictitle} \vspace{3mm}\\
		
		\begin{itemize}
			
			\item \sectiontitleI    \vspc % Section I
			\item \sectiontitleII 	\vspc % Section II
			\item \sectiontitleIII 	\vspc %Section III
			\item \sectiontitleIV 	\vspc %Section IV
			%\item \sectiontitleV 	\vspc %Section V
			
		\end{itemize}
		
	}
	

\section{\sectiontitleI}

\frame{
\frametitle{\sectiontitleI}

 \textbf{The Laplace Transform is an Integral Transform: } \\

Given a function $x(t)$ in the time domain where $t\geq0$, \\ the Laplace Transform is defined as follows: \\

\[ X(s)=\Lagr{\{x(t)\}}=\displaystyle\int_0^{\infty} x(t)e^{-st}dt \]

Partial Fraction Expansion leads to a general form:  \\

\[ X(s)=\frac{N(s)}{D(s)}=\frac{b_ms^m+b_{m-1}+...+b_1s+b_0}{s^n+a_{n-1}s^{n-1}+...+a_1s+a_0 \hspace{5mm} n\geq m}\] 


}


\section{\sectiontitleII}
\frame{
\frametitle{\sectiontitleI}

\textbf{Case 1 - Distinct Roots: n roots are real and distinct} 

	The general form is factored:
	\[ X(s)=\frac{N(s)}{(s+r_1)(s+r_2)...(s+r_n)} \]

	The fraction will expand to: 
	\[ X(s)=\frac{C_1}{(s+r_1)}+\frac{C_2}{(s+r_2)}+...+\frac{C_n}{(s+r_n)} \]  
	
	Where:
	\[ \displaystyle C_i=\lim_{s\rightarrow-r_i}\{ X(s)(s+r_i) \} \]

	And this leads to a solution: 
	\[x(t)=C_1e^{-r_1t}+C_2e^{-r_2t}+...+C_ne^{-r_nt} \]

}


\section{\sectiontitleIII}
\frame{ \small
\frametitle{\sectiontitleIII}

\textbf{Case 2 - Repeated Roots: p number of roots have the same value $(s =-r)$ and remaining roots are distinct and real distinct} 
	
	\[X(s)=\frac{N(s)}{(s+r_1)^p(s+r_{p+1})(s+r_{p+2})...(s+r_n)} \]

	The fraction will expand to: \\

	\[X(s)=\frac{C_1}{(s+r_1)^p}+\frac{C_2}{(s+r_1)^{p-1}}+...\] 
\[ +\frac{C_p}{(s+r_1)}+\frac{C_{p+1}}{(s+r_{p+1})}+...+\frac{C_n}{(s+r_n)} \]

}

\frame{ \small
\frametitle{\sectiontitleIII}

Coefficients for the repeated root are:\\

	\[\displaystyle C_1=\lim_{s\rightarrow-r_i}\{X(s)(s+r_i)^p\} \]
	\[\displaystyle C_2=\lim_{s\rightarrow-r_i}\{\frac{d}{ds}X(s)(s+r_i)^p\} \]
	
\[ \displaystyle C_i=\lim_{s\rightarrow-r_i}\{\frac{1}{(i-1)!}\frac{d^{(i-1)}}{ds^{(i-1)}}X(s)(s+r_i)^p\} \]

	Coefficients for the distinct roots are the same as in Case 1:\\

	And this leads to a solution: \\

	\[ x(t)=C_1\frac{t^{p-1}}{(p-1)!}e^{-r_1t}+C_2\frac{t^{p-2}}{(p-2)!}e^{-r_1t}+...\]
\[ ...+C_{p}e^{-r_1t}+C_{p+1}e^{-r_{p+1}t}...+C_ne^{-r_nt} \]

}

\section{\sectiontitleIV}
\frame{ \small
\frametitle{\sectiontitleIV}

\textbf{Special Case - Complex Roots: the roots are distinct $\implies$ Case 1} 

Example:

\[ X(s)=\left[ \frac{3s+7}{\left(4s^2+24s+136\right)} \right]=\left[ \frac{3s+7}{4\left(s^2+6s+34\right)} \right] \]

The solution can be found by forming two perfect squares in the denominator.

\[ X(s)=\frac{1}{4} \left[ \frac{3s+7}{\left(s+3\right)^2+5^2}\right] \]


}

\frame{ \small
\frametitle{\sectiontitleIV}

Now this can be expanded into the following terms which can be found in the table!

\[ X(s)=\frac{1}{4}  \left[ C_1 \frac{5}{\left(s+3\right)^2+5^2} +C_2 \frac{s+3}{\left(s+3\right)^2+5^2}\right] \]

Multiply by the denominator and solve for $C_1$ and $C_2$.

\[ 3s+7 = 5C_1 +C_2\left( s+3\right) = 5C_1 +c_2s+3C_2 \implies C_2=3, C_1 = -\frac{2}{5} \]

}

\frame{ \small
\frametitle{\sectiontitleIV}

Finally substitute and invert using the table. 

\[ X(s)=\frac{1}{4}  \left[-\frac{2}{5}\frac{5}{\left(s+3\right)^2+s^2} +3 \frac{s+3}{\left(s+3\right)^2+s^2}\right] \] 

Write the final answer in the time domain.
\[ x(t) =-\frac{1}{10}e^{-3t}sin\left(5t\right)+\frac{3}{4}e^{-3t}cos\left(5t\right) \]


}
\end{document}

