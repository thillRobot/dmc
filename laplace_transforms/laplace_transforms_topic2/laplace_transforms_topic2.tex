% Lecture Template for ME3050-001-002-Tristan Hill - Spring 2020
% Dynamics Modeling and Controls
% Second Order Time Response - Module 12 - Topic 4

% I am finally converting my stuff to BEAMER

% Document settings

%\documentclass{beamer}                  % for presentation ?
\documentclass[handout]{beamer}  % for handout ?
\usepackage{beamerthemesplit}
\usepackage{amsmath}
\usepackage{listings}
\usepackage{multicol}
\usepackage{framed}

\beamertemplateballitem

\definecolor{TTUpurple}{rgb}{0.3098, 0.1607, 0.5176} % TTU Purple (primary)
\definecolor{TTUgold}{rgb}{1.0000, 0.8666, 0.0000} % TTU Gold (primary)

\setbeamercolor{palette primary}{bg=TTUpurple,fg=TTUgold}
\setbeamercolor{palette secondary}{bg=black,fg=TTUgold}
\setbeamercolor{palette tertiary}{bg=black,fg=TTUpurple}
\setbeamercolor{palette quaternary}{bg=TTUgold,fg=black}
\setbeamercolor{structure}{fg=TTUpurple} % itemize, enumerate, etc
\setbeamercolor{section in toc}{fg=TTUpurple} % TOC sections

%\usefonttheme{professionalfonts}

\newcommand{\Lagr}{\mathcal{L}} % lagrangian

\newcommand{\hspcu}{\underline{\hspace{20mm}}} % large horizontal space w underline
\newcommand{\vspccc}{\vspace{6mm}\\} % large vertical space
\newcommand{\vspcc}{\vspace{4mm}\\}   % medium vertical space
\newcommand{\vspc}{\vspace{2mm}\\}     % small vertical space

\newcommand{\hspcccc}{\hspace{10mm}} % large horizontal space
\newcommand{\hspccc}{\hspace{6mm}} % large horizontal space
\newcommand{\hspcc}{\hspace{4mm}}   % medium horizontal space
\newcommand{\hspc}{\hspace{2mm}}     % small horizontal space

\newsavebox{\mybox} % custom box

\newcommand{\MNUM}{10\hspace{2mm}} % Module number
\newcommand{\TNUM}{2\hspace{2mm}} % Topic number 
\newcommand{\moduletitle}{The Laplace Transform} % Titles and Stuff
\newcommand{\topictitle}{Laplace Transforms Method} 

\newcommand{\sectiontitleI}{Step 1 - Apply Laplace Transform} % More Titles and Stuff
\newcommand{\sectiontitleII}{Step 2 - Solve for $X(s)$}
\newcommand{\sectiontitleIII}{Step 3 - Rearrange to Find Invertable Form}
\newcommand{\sectiontitleIV}{Step 4 - Invert for Final Answer}

\author{ME3050 - Dynamics Modeling and Controls}
\title{Module \MNUM - \moduletitle}
\date{Mechanical Engineering\vspc Tennessee Technological University}

\begin{document}

\lstset{language=MATLAB,basicstyle=\ttfamily\small,showstringspaces=false}

\frame{\titlepage \center\begin{framed}\Large \textbf{Topic \TNUM - \topictitle}\end{framed} \vspace{5mm}}

% Section 0 - Outline
\frame{
	
	\large \textbf{Topic \TNUM - \topictitle} \vspace{3mm}\\
	
	\begin{itemize}
	
		\item \sectiontitleI    \vspc % Section I
		\item \sectiontitleII 	\vspc % Section II
		\item \sectiontitleIII 	\vspc %Section III
		\item \sectiontitleIV 	\vspc %Section IV
	
	\end{itemize}

}


\section{\sectiontitleI}

\frame{
\frametitle{\sectiontitleI}

Example:

Solve the first order differential equation using the Laplace Transforms Method with the initial condition given. 
\[4\dot{x}=sin\left(t\right) \hspccc with \hspccc x\left(t=0\right)=x_0 \] 
Apply the Laplace Transform to both sides of the differential equation. 

\[ 4\left(sX\left(s \right)-x_0 \right)=\frac{1}{s^2+1} \]

}

\section{\sectiontitleII}
\frame{
\frametitle{\sectiontitleII}
This step can seem open ended...
\[X(s)=\frac{1}{4s\left(s^2+1\right)}+\frac{x_0}{s} \] 

}

\section{\sectiontitleIII}
\frame{
\frametitle{\sectiontitleIII}

Write $X\left(s\right)$ in a form that can be inverted using the table of Laplace transform pairs. This typically involves partial fraction decomposition. 
\[\frac{1}{4s\left(s^2+1 \right)}=\frac{1/4}{s\left(s^2+1\right)} =\frac{a}{s}+\frac{bs+c}{s^2+1} \] 

Mulitply through by the denominator $4s\left(s^2+1\right)$:
\[ 1=4as\left(s^2+1\right)+4s\left(bs+c\right)=4\left(a+b \right)s^2 + 4cs +4a \]

Solve for the coefficients by {\it equating coefficients}.

\[ \left( a+b\right)=0 \hspcc c=0 \hspcc a=\frac{1}{4} \implies a=\frac{1}{4} \hspcc b=-\frac{1}{4} \hspcc c=0 \]
}

\section{\sectiontitleIV}
\frame{
\frametitle{\sectiontitleIV}

Substitute the coefficients into $X(s)$,

\[ X(s) =\frac{x_0}{s}+\frac{1}{4s}-\frac{s}{4\left(s^2+1\right)} \]

and use the inverse transform to solve for $x(t)$. Use the Table.

\[ \Lagr^{-1} \left( X\left( s\right)\right)=x(t)= \]
\[ = x_0+\frac{1}{4}-\frac{1}{4}cos\left(t\right)=x_0+\frac{1}{4}\left(1-cos\left(t\right) \right) \]

This method works for complex problems but it can get messy...
}

\end{document}

