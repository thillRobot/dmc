% Lecture Template for ME3001-001-Tristan Hill - Spring 2017 - Fall 2017
% 
% Mechanical Engineering Analysis with MATLAB
%
% Introduction to Analysis

% Document settings
\documentclass[11pt]{article}
\usepackage[margin=1in]{geometry}
\usepackage[pdftex]{graphicx}
\usepackage{multirow}
\usepackage{setspace}
\usepackage{hyperref}
\usepackage{color,soul}
\usepackage{fancyvrb}
\usepackage{framed}
\usepackage{wasysym}
\usepackage{multicol}
\usepackage{ amssymb }



\pagestyle{plain}
\setlength\parindent{0pt}
\hypersetup{
    bookmarks=true,         % show bookmarks bar?
    unicode=false,          % non-Latin characters in Acrobat’s bookmarks
    pdftoolbar=true,        % show Acrobat’s toolbar?
    pdfmenubar=true,        % show Acrobat’s menu?
    pdffitwindow=false,     % window fit to page when opened
    pdfstartview={FitH},    % fits the width of the page to the window
    pdftitle={My title},    % title
    pdfauthor={Author},     % author
    pdfsubject={Subject},   % subject of the document
    pdfcreator={Creator},   % creator of the document
    pdfproducer={Producer}, % producer of the document
    pdfkeywords={keyword1} {key2} {key3}, % list of keywords
    pdfnewwindow=true,      % links in new window
    colorlinks=true,       % false: boxed links; true: colored links
    linkcolor=red,          % color of internal links (change box color with linkbordercolor)
    citecolor=green,        % color of links to bibliography
    filecolor=magenta,      % color of file links
    urlcolor=blue           % color of external links
}

% assignment number 
\newcommand{\NUM}{1 } 
\newcommand{\VSpaceSize}{2mm} 
\newcommand{\HSpaceSize}{2mm} 

\newcommand{\Lagr}{\mathcal{L}}

\definecolor{mygray}{rgb}{.6, .6, .6}

\setulcolor{red} 
\setstcolor{green} 
\sethlcolor{mygray} 

\begin{document}

\textbf{ \LARGE ME 3050 Lecture - Laplace Transforms} \vspace{3mm}\\
\textbf{ \hspace*{5mm}Tristan W. Hill - Tennessee Technological University - Spring 2020 } \vspace{3mm}\\

\Large
\begin{itemize}

\item \textbf{ \LARGE Partial Fraction Expansion leads to a General Form: } \\

\scalebox{1.5}{$X(s)=\frac{N(s)}{D(s)}=\frac{b_ms^m+b_{m-1}+...+b_1s+b_0}{s^n+a_{n-1}s^{n-1}+...+a_1s+a_0}$\hspace{5mm}$n\geq m$} \vspace{5mm}\\

\item \textbf{ \LARGE Case 1 - Distinct Roots: n roots are real and distinct} \vspace{5mm}\\
\Large

	The general form is factored:\\
	
	\scalebox{1.5}{$X(s)=\frac{N(s)}{(s+r_1)(s+r_2)...(s+r_n)}$} \vspace{5mm}\\

	The fraction will expand to: \\

	\scalebox{1.5}{$X(s)=\frac{C_1}{(s+r_1)}+\frac{C_2}{(s+r_2)}+...+\frac{C_n}{(s+r_n)}$} \vspace{5mm}\\

	Where:\\

	\scalebox{1.5}{$\displaystyle C_i=\lim_{s\rightarrow-r_i}\{ X(s)(s+r_i) \} $} \vspace{5mm}\\

	And this leads to a solution: \\

	\scalebox{1.5}{$x(t)=C_1e^{-r_1t}+C_2e^{-r_2t}+...+C_ne^{-r_nt}$}

\newpage
\item \textbf{ \LARGE Case 2 - Repeated Roots: \\ p number of roots have the same value $(s =-r)$ and remaining roots are distinct and real distinct} \vspace{5mm}\\
\Large
	
	
	\scalebox{1.25}{$X(s)=\frac{N(s)}{(s+r_1)^p(s+r_{p+1})(s+r_{p+2})...(s+r_n)}$} \vspace{5mm}\\

	The fraction will expand to: \\

	\scalebox{1.25}{$X(s)=\frac{C_1}{(s+r_1)^p}+\frac{C_2}{(s+r_1)^{p-1}}+...$} \vspace{5mm}\\
\scalebox{1.25}{$ +\frac{C_p}{(s+r_1)}+\frac{C_{p+1}}{(s+r_{p+1})}+...+\frac{C_n}{(s+r_n)}$} \vspace{5mm}\\

	Coefficients for the repeated root are:\\

	\scalebox{1.25}{$\displaystyle C_1=\lim_{s\rightarrow-r_i}\{X(s)(s+r_i)^p\}$} \vspace{5mm}\\
	\scalebox{1.25}{$\displaystyle C_2=\lim_{s\rightarrow-r_i}\{\frac{d}{ds}X(s)(s+r_i)^p\}$} \vspace{5mm}\\
\scalebox{1.25}{$\displaystyle C_i=\lim_{s\rightarrow-r_i}\{\frac{1}{(i-1)!}\frac{d^{(i-1)}}{ds^{(i-1)}}X(s)(s+r_i)^p\}$} \vspace{5mm}\\

	Coefficients for the distinct roots are the same as in Case 1:\\

	And this leads to a solution: \\

	\scalebox{1.5}{$x(t)=C_1\frac{t^{p-1}}{(p-1)!}e^{-r_1t}+C_2\frac{t^{p-2}}{(p-2)!}e^{-r_1t}+...$} \vspace{5mm}\\
\scalebox{1.5}{$...+C_{p}e^{-r_1t}+C_{p+1}e^{-r_{p+1}t}...+C_ne^{-r_nt}$}




\end{itemize}

\end{document}



