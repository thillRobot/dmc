% Lecture Template for ME3050-001-Tristan Hill - Spring 2020
% 
% Dynamics Modeling and Controls
% Dyanmics Review


% Document settings
\documentclass[11pt]{article}
\usepackage[margin=1in]{geometry}
\usepackage[pdftex]{graphicx}
\usepackage{multirow}
\usepackage{setspace}
\usepackage{hyperref}
\usepackage{color,soul}
\usepackage{fancyvrb}
\usepackage{framed}
\usepackage{wasysym}
\usepackage{multicol}

\pagestyle{plain}
\setlength\parindent{0pt}
\hypersetup{
    bookmarks=true,         % show bookmarks bar?
    unicode=false,          % non-Latin characters in Acrobat’s bookmarks
    pdftoolbar=true,        % show Acrobat’s toolbar?
    pdfmenubar=true,        % show Acrobat’s menu?
    pdffitwindow=false,     % window fit to page when opened
    pdfstartview={FitH},    % fits the width of the page to the window
    pdftitle={My title},    % title
    pdfauthor={Author},     % author
    pdfsubject={Subject},   % subject of the document
    pdfcreator={Creator},   % creator of the document
    pdfproducer={Producer}, % producer of the document
    pdfkeywords={keyword1} {key2} {key3}, % list of keywords
    pdfnewwindow=true,      % links in new window
    colorlinks=true,       % false: boxed links; true: colored links
    linkcolor=red,          % color of internal links (change box color with linkbordercolor)
    citecolor=green,        % color of links to bibliography
    filecolor=magenta,      % color of file links
    urlcolor=blue           % color of external links
}

% assignment number 
\newcommand{\NUM}{3 } 
\newcommand{\VSpaceSize}{2mm} 
\newcommand{\HSpaceSize}{2mm} 

\newcommand{\B}{\color{blue}} 
\newcommand{\K}{\color{black}}
%\newcommand{\B}{\color{blue}} 

\definecolor{mygray}{rgb}{.6, .6, .6}

\setulcolor{red} 
\setstcolor{green} 
\sethlcolor{mygray} 

\begin{document}

\textbf{ \LARGE ME 3050 Lecture - Dynamics Review - Lecture \NUM  } \\

\begin{itemize}



	\item \textbf{ \Large Motion can be represented as a combination of \vspace{3mm}\\
\\\\ \underline{\hspace{60mm}} and \underline{\hspace{60mm}}.} \vspace{2mm}\\

\item \textbf{ \Large All of the points on a single rigid body in motion  \vspace{5mm}\\have the same \underline{\hspace{50mm}}.} \vspace{5mm}\\


	\item \textbf{ \Large Today we are going to talk about the dynamics of \B Rotation\K. } \\
\Large
	\begin{enumerate}
		\item \textbf{ Rotation About a \B Fixed Axis \K (Ch3.2)} \vspace{0mm} \\
		\begin{itemize}
		\item Moment Equations
		\item Calculating Inertias
		\item Parallel Axis Theorem\\
		\end{itemize}
		\item  \textbf{ \B Sliding \K and \B Rolling \K Motion (Ch3.3.2)} \vspace{3mm} \\
	
		\item  \textbf{ \B General Planar \K Motion (Ch3.4)}\vspace{3mm} \\
	
	\end{enumerate}

\newpage


	\begin{enumerate}
		\item \textbf{ Rotation About a \B Fixed Axis \K} \vspace{3mm} \\
		\newpage
		\item  \textbf{ \B Sliding \K and \B Rolling \K Motion} \vspace{3mm} \\
		\newpage
		\item  \textbf{ \B General Planar \K Motion} \vspace{3mm} \\
	
	\end{enumerate}




\end{itemize}


	

\end{document}



