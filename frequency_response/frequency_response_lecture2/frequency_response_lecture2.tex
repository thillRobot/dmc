% Lecture Template for ME3050-001-002-Tristan Hill - Spring 2020
% Dynamics Modeling and Controls
% Frequency Response

% I am finally converting my stuff to BEAMER

% Document settings

%\documentclass{beamer}                  % for presentation ?
\documentclass[handout]{beamer}  % for handout ?
\usepackage{beamerthemesplit}
\usepackage{amsmath}
\usepackage{listings}
\usepackage{multicol}
\usepackage{framed}
\usepackage{amssymb}


\lstdefinestyle{myCustomMatlabStyle}{
  language=Matlab,
  numbers=left,
  stepnumber=1,
  numbersep=10pt,
  tabsize=4,
  showspaces=false,
  showstringspaces=false
}
\lstset{basicstyle=\ttfamily\tiny,style=myCustomMatlabStyle}
%lstset{language=MATLAB,basicstyle=\ttfamily\small,showstringspaces=false}



\beamertemplateballitem

\definecolor{TTUpurple}{rgb}{0.3098, 0.1607, 0.5176} % TTU Purple (primary)
\definecolor{TTUgold}{rgb}{1.0000, 0.8666, 0.0000} % TTU Gold (primary)

\setbeamercolor{palette primary}{bg=TTUpurple,fg=TTUgold}
\setbeamercolor{palette secondary}{bg=black,fg=TTUgold}
\setbeamercolor{palette tertiary}{bg=black,fg=TTUpurple}
\setbeamercolor{palette quaternary}{bg=TTUgold,fg=black}
\setbeamercolor{structure}{fg=TTUpurple} % itemize, enumerate, etc
\setbeamercolor{section in toc}{fg=TTUpurple} % TOC sections

%\DeclareSymbolFont{bbold}{U}{bbold}{m}{n}
%\DeclareSymbolFontAlphabet{\mathbbold}{bbold}

%\newcommand{\bbfamily}{\fontencoding{U}\fontfamily{bbold}\selectfont}
%\DeclareMathAlphabet{\mathbbold}{U}{bbold}{m}{n}

%\usefonttheme{professionalfonts}

\newcommand{\vspccc}{\vspace{6mm}\\} % large vertical space
\newcommand{\vspcc}{\vspace{4mm}\\}   % medium vertical space
\newcommand{\vspc}{\vspace{2mm}\\}     % small vertical space

\newcommand{\hspcccc}{\hspace{10mm}} % large horizontal space
\newcommand{\hspccc}{\hspace{6mm}} % large horizontal space
\newcommand{\hspcc}{\hspace{4mm}}   % medium horizontal space
\newcommand{\hspc}{\hspace{2mm}}     % small horizontal space


\newcommand{\LT}{\mathcal{L}} % lagrangian

\newcommand{\LNUM}{2 } %Lecture number 2

\newcommand{\secondtitle}{The Bode Diagram}% second line of the title of this presentation , aka the topic of this lecture

\title{Frequency Response - Lecture \LNUM}
\author{ME3050 - Dynamics Modeling and Controls} % original formatting from Mike Renfro, September 21, 2004

\date{April  19, 2020}

\begin{document}

\lstset{language=MATLAB,basicstyle=\ttfamily\small,showstringspaces=false}

% Title page1 
\frame{\titlepage \center\textbf{\secondtitle}\vspcc}


% Section 0: Outline
\frame{

\large \textbf{Lecture \LNUM - \secondtitle} \vspc

 \begin{itemize}

	\item Review Frequency Response\vspc % Section 1: 

	\item Magnitude Ratio in Decibels\vspc % Section 2
	
	\item The Bode Diagram\vspc %Section 3

	\item Graph of Frequency Response in MATLAB\vspace{2mm} % Section 4

\end{itemize}

}


%Section 1: Review Frequency Response
\section{Review Frequency Response}

\subsection{Harmonic Input Function}
\frame{
\frametitle{Harmonic Input Function}

\small 
The term {\bf frequency response} is used to describe a system's response to a periodic input. Frequency response analysis focuses on a system's response to {\it harmonic} input such as sines and cosines. The input (forcing) function is written below.\vspcc

\begin{framed}
\scalebox{1.25}{$f(t)=Asin\left(\omega t\right)$}\vspccc

\renewcommand{\arraystretch}{1.5}
\begin{tabular}{ccc}
Amplitude of the Input, & \scalebox{1}{$A$} & \scalebox{1}{$ (N)$} \\
Frequency of Input, &\scalebox{1}{$\omega$} &  \scalebox{1}{$(\frac{rad}{s})$}\\
\end{tabular}
\end{framed}

}

\subsection{First Order Frequency Response}
\frame{
\frametitle{First Order Frequency Response}


\small


The steady state response we derived is shown. Remember, after some amount of time passes, the transient term will disappear leaving just the sinusoidal terms. \vspc
\begin{framed}
\scalebox{1}{$y_{ss}\left(t\right)=A|T\left(j\omega\right)|sin\left(\omega t+\angle T\left(j\omega\right)\right)=MAsin\left(\omega t+\phi\right)$}\vspcc

The magintude ratio and phase shift can be found from $T(j\omega)$. \vspcc

\scalebox{1}{$M(\omega)=|T(j\omega)|=\frac{1}{\sqrt{1+\tau^2\omega^2}}$}\vspc

\scalebox{1}{$\phi(\omega)=\angle T\left(j\omega\right)=-tan^{-1}\left(\omega\tau\right)$}\vspc
\end{framed}

}


\subsection{Graph of Frequency Response}
\frame{
\frametitle{Graph of Frequency Response}

\small

\includegraphics[scale=.275]{lecture1_fig6.png} \hspc \includegraphics[scale=.275]{lecture1_fig5.png}  \vspc

The amplitude of the response is determined by the input frequency. \vspc

}


% section 2: Magnitude Ratio in Decibels

\section{Magnitude Ratio in Decibels}

\subsection{Dependence on Input Frequency}
\frame{
\frametitle{Dependence on Input Frequency}
\begin{multicols}{2}

\includegraphics[scale=.27]{lecture2_fig2.png}

You can see that the magnitude ratio decreases as the input frequency increases. The indivdual curves represent systems different time constants. 

\end{multicols}

}

\subsection{Review Properties of Logarithms}
\frame{
\frametitle{Review Properties of Logarithms}

\underline{Basic Properties of Logarithms:}\vspc
\renewcommand{\arraystretch}{2.0}
\begin{tabular}{cc}
Multiplication&\scalebox{1}{$log(pq)=log(p)+log(q) $} \\
Division&\scalebox{1}{$log(\frac{x}{y})=log(x)-log(y)$} \\
Power&\scalebox{1}{$log(x^n)=nlog(x)$}\\
\end{tabular}

\underline{Units of Decibels for Magnitude:}\vspc
\scalebox{1}{$m(dB)=10log(M^2)=20log(M)$ \hspccc convert back: \hspccc $M=10^{\frac{m(dB)}{20}}$}
}

\section{Magnitude Ratio in Decibels}

\subsection{Magnitude Ratio on a Logarithmic Scale}
\frame{
\frametitle{Magnitude Ratio on a Logarithmic Scale}


These relationships are more useful shown on a logarithmic scale. We can make use of the properties of logorithms in our analysis. \vspcc

\scalebox{1}{$m\left(dB\right)=20log\left(\frac{1}{\sqrt{1+\omega^2\tau^2}}\right)=20\left(log\left(1\right)-log\sqrt{1+\omega^2\tau^2}\right)$}\vspc
\scalebox{1}{$m\left(dB\right)=20log\left(1\right)-10log\left(1+\omega^2\tau^2\right)=-10log\left(1+\omega^2\tau^2\right)$}\vspc
\begin{framed}
\scalebox{1}{$m\left(dB\right)=-10log\left(1+\omega^2\tau^2\right)$}\vspc magnitude ratio in decibels
\end{framed}
}

\subsection{Magnitude Ratio on a Logarithmic Scale}
\frame{
\frametitle{Magnitude Ratio on a Logarithmic Scale}
\begin{multicols}{2}
This is called a Bode plot.  \includegraphics[scale=.5]{hendrikbode.png} \vspc Hendrik Bode (1905-1982) 
\includegraphics[scale=.25]{lecture2_fig7.png}
\end{multicols}
}

% references is not a section for now, for looks and it would be a waste of space
\frame{

\frametitle{References}

\begin{itemize}
	\item System Dynamics, Palm III, Third Edition - Chapter 9 - System Response in the Frequency Domain
\end{itemize}

}

\end{document}









 