% Lecture Template for ME3050-001-002-Tristan Hill - Spring 2020 - Summer
% Dynamics Modeling and Controls
% Higher Order Systems - Modeul 13 - Topic 2

% I am finally converting my stuff to BEAMER

% Document settings

%\documentclass{beamer}                  % for presentation ?
\documentclass[handout]{beamer}  % for handout ?
\usepackage{beamerthemesplit}
\usepackage{amsmath}
\usepackage{listings}
\usepackage{multicol}
\usepackage{framed}
\usepackage{amsmath, nccmath}
\usepackage{geometry}
\usepackage{bm}

\beamertemplateballitem

\definecolor{TTUpurple}{rgb}{0.3098, 0.1607, 0.5176} % TTU Purple (primary)
\definecolor{TTUgold}{rgb}{1.0000, 0.8666, 0.0000} % TTU Gold (primary)

\setbeamercolor{palette primary}{bg=TTUpurple,fg=TTUgold}
\setbeamercolor{palette secondary}{bg=black,fg=TTUgold}
\setbeamercolor{palette tertiary}{bg=black,fg=TTUpurple}
\setbeamercolor{palette quaternary}{bg=TTUgold,fg=black}
\setbeamercolor{structure}{fg=TTUpurple} % itemize, enumerate, etc
\setbeamercolor{section in toc}{fg=TTUpurple} % TOC sections

% custom colors
\definecolor{TTUpurple}{rgb}{0.3098, 0.1607, 0.5176} % TTU Purple (primary)
\definecolor{TTUgold}{rgb}{1.0000, 0.8666, 0.0000} % TTU Gold (primary) 
\definecolor{mygray}{rgb}{.6, .6, .6}
\definecolor{mypurple}{rgb}{0.6,0.1961,0.8}
\definecolor{mybrown}{rgb}{0.5451,0.2706,0.0745}
\definecolor{mygreen}{rgb}{0, .39, 0}
\definecolor{mypink}{rgb}{0.9960, 0, 0.9960}

% color commands
\newcommand{\R}{\color{red}}
\newcommand{\B}{\color{blue}}
\newcommand{\BR}{\color{mybrown}}
\newcommand{\K}{\color{black}}
\newcommand{\G}{\color{mygreen}}
\newcommand{\PR}{\color{mypurple}}
\newcommand{\PN}{\color{mypink}}
\newcommand{\OR}{\color{orange}}
\newcommand{\GD}{\color{TTUgold}}


\newcommand{\Lagr}{\mathcal{L}} % lagrangian

\newcommand{\hspcu}{\underline{\hspace{20mm}}} % large horizontal space w underline
\newcommand{\vspccc}{\vspace{6mm}\\} % large vertical space
\newcommand{\vspcc}{\vspace{4mm}\\}   % medium vertical space
\newcommand{\vspc}{\vspace{2mm}\\}     % small vertical space

\newcommand{\hspcccc}{\hspace{10mm}} % large horizontal space
\newcommand{\hspccc}{\hspace{6mm}} % large horizontal space
\newcommand{\hspcc}{\hspace{4mm}}   % medium horizontal space
\newcommand{\hspc}{\hspace{2mm}}     % small horizontal space

\newsavebox{\mybox} % custom box

\newcommand{\MNUM}{13\hspace{2mm}} % Module number
\newcommand{\TNUM}{3\hspace{2mm}} % Topic number 
\newcommand{\moduletitle}{Alternate Model Forms} % Titles and Stuff
\newcommand{\topictitle}{State Space Representations} 

\newcommand{\sectiontitleI}{State Space Models} % More Titles and Stuff
\newcommand{\sectiontitleII}{The State Space Equation}
\newcommand{\sectiontitleIII}{The Output Equation}
\newcommand{\sectiontitleIV}{Simulation and Control Applications}

\author{ME3050 - Dynamics Modeling and Controls}
\title{Lecture Module - \moduletitle}
\date{Mechanical Engineering\vspc Tennessee Technological University}

\begin{document}

\lstset{language=MATLAB,basicstyle=\ttfamily\small,showstringspaces=false}

\frame{\titlepage \center\begin{framed}\Large \textbf{Topic \TNUM - \topictitle}\end{framed} \vspace{5mm}}

% Section 0 - Outline
\frame{
	
	\large \textbf{Topic \TNUM - \topictitle} \vspace{3mm}\\
	
	\begin{itemize}
	
		\item \sectiontitleI    \vspc % Section I
		\item \sectiontitleII 	\vspc % Section II
		\item \sectiontitleIII 	\vspc %Section III
		\item \sectiontitleIV 	\vspc %Section IV
	
	\end{itemize}

}


\section{\sectiontitleI}

\frame{
  \frametitle{\sectiontitleI}
    \begin{itemize}

		\item Most linear differential equations can be written in state space form.\\

		\item A state space model is an equivalent representation of a dynamic system.\\
		
		\item This standard form allows us to use and share tools for analysis and design of complex systems. 

	\end{itemize}  

}


\frame{
  	\frametitle{\sectiontitleI}
	\textbf{ Higher Order Differential Equations} - An $N^{th}$ order differential equation can be {\PN decomposed} into a system of $N$ first order differential equations. The resulting system is equivalent to the original equation. \vspace{30mm}

}

\section{\sectiontitleII}

\frame{
  	\frametitle{\sectiontitleII}
	After the system of differential equations consists of first order equations only, these equations form {\B the state equation}.  

}


\frame{
  	\frametitle{\sectiontitleII}
	\textbf{The State Equation} 

	\[ \bm{ \dot{x} } = \bm{Ax} + \bm{Bu} \]

	\begin{itemize}
		\item there are n {\B state variables} or {\B states}:  $x_1 - x_n$ 
		\item there are m {\G inputs} called $u_1 - u_m$ 
		\item the {\B state vector} $\bm{x}$ is a collumn vector with $n$ rows
		\item the {\R system matrix} $\bm{A}$ is a square matrix $n$ rows and {\it n} columns.
		\item the {\PR input vector} $\bm{u}$ is a column vector with $m$ rows.
		\item the {\PN control} or input matrix $\bm{B}$ is a matrix with $n$ rows and {\it m} columns. 
	\end{itemize}		
		
}



\section{\sectiontitleIII}

\frame{
  	\frametitle{\sectiontitleIII}
	\textbf{The Output Equation} 

	\[ \bm{y} = \bm{Cx} + \bm{Du} \]

	\begin{itemize}
		\item the {\OR output vector} $\bm{y}$ is a collumn vector with {\it p} rows
		\item the {\G output matrix} $\bm{C}$ is a square matrix {\it p} rows and {\it n} columns.
		\item the {\PN control matrix} $\bm{D}$ is a matrix with {\it p} rows and {\it m} columns.
	\end{itemize}

\vspace{5mm}
	The designer chooses any combination of dependent variables or derived quantities for the output equation for your individual purposes. The number of outputs is {\it flexible}. \vspc
	  
}

\frame{  
	\frametitle{\sectiontitleIII}
	Choose the outputs that you want to study and write the output equations as functions of the states and {\bf not} their derivatives.  	  

}

\section{\sectiontitleIV}


\frame{  
  \frametitle{\sectiontitleIV}

    \begin{itemize}

		\item {\it commonly used} for system models \vspc 
		
		\item useful for {\it numerical simulation}\vspc

		\item used in the area of {\it automatic control}\vspc

		\item This standard form allows us to use and share tools for analysis and design of complex systems.  \vspc

	\end{itemize}

}

% references is not a section for now, for looks and it would be a waste of space
\frame{

	\frametitle{References}
	
	\begin{itemize}
		\item System Dynamics, Palm III, Third Edition - Chapter 4 - Spring and Damper Elements in Mechanical Systems - pg. 208
	\end{itemize}

}

\end{document}









