% 
% Lecture Template for ME3050 -  Dynamics Modeling and Controls - Tennessee Technological University
%
% Spring 2020 - Summer 2020
% Tristan Hill, May 07, 2020
% Dyanmics Review - Topic 5 - Coordinate Systems
%

\documentclass{beamer}                         % for presentation (has nav buttons at bottom)
%\documentclass[handout]{beamer}  % for handout 
\usepackage{beamerthemesplit}
\usepackage{amsmath}
\usepackage{listings}
\usepackage{multicol}

\beamertemplateballitem

\definecolor{TTUpurple}{rgb}{0.3098, 0.1607, 0.5176} % TTU Purple (primary)
\definecolor{TTUgold}{rgb}{1.0000, 0.8666, 0.0000} % TTU Gold (primary)

\setbeamercolor{palette primary}{bg=TTUpurple,fg=TTUgold}
\setbeamercolor{palette secondary}{bg=black,fg=TTUgold}
\setbeamercolor{palette tertiary}{bg=black,fg=TTUpurple}
\setbeamercolor{palette quaternary}{bg=TTUgold,fg=black}
\setbeamercolor{structure}{fg=TTUpurple} % itemize, enumerate, etc
\setbeamercolor{section in toc}{fg=TTUpurple} % TOC sections

%\usefonttheme{professionalfonts}

\newcommand{\Lagr}{\mathcal{L}} % lagrangian

\newcommand{\vspccc}{\vspace{6mm}\\} % large vertical space
\newcommand{\vspcc}{\vspace{4mm}\\}   % medium vertical space
\newcommand{\vspc}{\vspace{2mm}\\}     % small vertical space

\newcommand{\hspcccc}{\hspace{10mm}} % large horizontal space
\newcommand{\hspccc}{\hspace{6mm}} % large horizontal space
\newcommand{\hspcc}{\hspace{4mm}}   % medium horizontal space
\newcommand{\hspc}{\hspace{2mm}}     % small horizontal space


\author{ME3050 - Dynamics Modeling and Controls\\Tennessee Technological University \\} % original formatting from Mike Renfro, September 21, 2004

\newcommand{\TNUM}{5\hspace{2mm}} % topic Number 
\newcommand{\topictitle}{Coordinate Systems } % first line of title (used by beamer)
\newcommand{\sectiontitle}{Dynamics Review }% second line of the title of this presentation (used by TWH)

\title{\sectiontitle - \topictitle}

\date{May 29, 2020}

\begin{document}

\lstset{language=MATLAB,basicstyle=\ttfamily\small,showstringspaces=false}

\frame{\titlepage \center\textbf{Topic \TNUM - \topictitle}\vspace{5mm}\\}

% Section 0: Outline


\frame{

\large \textbf{Topic \TNUM - \topictitle} \vspace{3mm}\\

%Topics : \vspace{3mm}\\ % ' topics' are beamer 'sections' - TWH

\begin{itemize}
	\item Using Different Coordinate Systems\vspace{3mm}\\ % Section 1
	\item Cartesian\vspace{3mm}\\% Section 2
	\item Circular or Cylindrical\vspace{3mm}\\ %Section 3
	\item Spherical \vspace{3mm}\\  %Section 4
	\item Others \vspace{3mm}\\  %Section 5
\end{itemize}
}

% Section 1:
\section{Using Different Coordinate Systems}

\frame{
\frametitle{Using Different Coordinate Systems}

It is often convienent to use different coordinate systems as a reference for different types of problems. \vspc

You, the engineer and designer must choose the coordinate system.

}

% Section 2: 
\section{Increase Compexity Incrementally}

\frame{
\frametitle{Increase Compexity Incrementally}

You cannot solve a complex problem in your head or all at once. \vspc

Engineers model and analyse complex systesm one peice at a time on a component level. \vspc

In system dynamics we study the system behavior by modeling the interations and responses of the different components involved.

}

% Section 3: 
\section{Solid Mechanics and Dynamics}

\frame{
\frametitle{Solid Mechanics and Dynamics}

\begin{itemize}
\item Frictionless Sliding
\item Pure Roll - No Slip
\item Planar Motion
\item
\end{itemize}

}

% Section 4:
\section{Thermal and Fluid Systems}

\frame{
\frametitle{Thermal and Fluid Systems}

\begin{itemize}
\item Viscous Boundary Layer
\item Insulated or Constant Flux Boundaries
\item
\end{itemize}

}

% Section 5:
\section{Electrical and Power Systems}

\frame{
\frametitle{Electrical and Power Systems}

\begin{itemize}
\item Zero Heat Loss or Generation
\item Zero Resistance Conductors
\item
\end{itemize}

}
	
\end{document}



